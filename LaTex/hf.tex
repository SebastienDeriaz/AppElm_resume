\documentclass[resume]{subfiles}


\begin{document}
\section{Hautes fréquences}
$$\SI{1}{\watt}=\SI{0}{\decibel}=\SI{30}{\decibel m}$$
$$\SI{1}{\milli\watt}=\SI{-30}{\decibel}=\SI{0}{\decibel m}$$
\section{Antennes}
$$\boxed{\lambda=\frac{c}{f}}$$
Vitesse de la lumière dans un milieu donné (avec $\mu_r$ et $\varepsilon_0$)
$$c_r=\frac{1}{\sqrt{\varepsilon\mu}}=\frac{1}{\sqrt{\varepsilon_r\varepsilon_0\mu_r\mu_0}}=\frac{c}{\sqrt{\mu_r\varepsilon_r}}$$

\subsection{Dipôle}
2 brins de longueur totale $l$ et de diamètre $a$
$$l\ll \lambda$$
$$a\ll\lambda$$
Il existe trois zones
\begin{itemize}
\item Champ proche (zone de champ proche non-rayonnant)
\item Zone de Fresnel (zone de champ proche rayonnant)
\item Zone de Fraunhofer (zone de champ lointain)
\end{itemize}
Il est important de se placer dans la bonne zone lorsqu'on fait des tests.
\subsection{Réflexion}
$$\Gamma=\frac{Z_{in}-Z_0}{Z_{in}+Z_0}$$
$$S_{11}=20\log_{10}(\Gamma)\quad [\si{\deci\bel}]$$
\subsubsection{Transmission}
$$\abs{\Gamma}^2+\abs{T}^2=1$$
\subsection{Rayonnement}
On utilise la formule de Friis
$$P_r=P_tG_rG_t\left(\frac{\lambda}{4\pi R}\right)^2$$
en dB :
$$P_r=P_t+G_r+G_t+20\log\left(\frac{\lambda}{4\pi R}\right)$$
\subsection{Puissance}
$$\boxed{P_{\si{\deci\bel}}=10\log_{10}\left(P_{\si{\watt}}\right)}$$
$$\boxed{P_{\si{\deci\bel m}}=10\log_{10}\left(P_{\si{\milli\watt}}\right)}$$
\subsection{Zones}
\paragraph{Zone de Fresnel} $r_1 < x < r_2$
\paragraph{Zone de Fraunhofer} $ r_2 < x$

$$r_1=\sqrt{0.38\frac{D^3}{\lambda}}$$
$$r_2=\frac{2D^2}{\lambda}$$
Avec $D$ la plus grand dimension de l'élément rayonnant (d'un bord à l'autre dans le cas d'une dipôle)
\subsection{Directivité}
$$D=\frac{4\pi}{\theta_{1r}\theta_{2r}}$$
Avec $\theta_i$ les angles d'ouverture à moitié de puissance respectement de 2 plans perpendiculaires.
\paragraph{Isotrope} : boule
\paragraph{Hémisphérique} : demi boule

\subsection{Paramètres antennes}
Valeurs importantes en gras.
\begin{enumerate}
\item Physique
	\begin{itemize}
	\item taille
	\item permittivité
	\item conductivité
	\end{itemize}
\item Circuit
	\begin{itemize}
	\item $S_{11}$ en fct de f
	\item $S_{11}$ à $f_{res}$
	\item \textbf{Fréquence de résonance}
	\item \textbf{Bande passante à $-10[dB]$}
	\end{itemize}
\item Rayonnement
	\begin{itemize}
	\item \textbf{gain}
	\item \textbf{efficacité}
	\item \textbf{directivité}
	\item diag de rayonnement
	\item angle d'ouverture à $-3[dB]$
	\end{itemize}
\end{enumerate}

\subsection{Remarques}
La différence entre le gain réalisé et le gain IEEE, le gain IEEE ne prend pas en compte d'éventuelles désadaptation de l'antenne (Hypothèse théorique d'une antenne parfaitement accordée). Le gain réalisé lui prend en compte cela (mesure réelle)





\end{document}