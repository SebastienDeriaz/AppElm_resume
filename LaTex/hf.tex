\documentclass[resume]{subfiles}


\begin{document}
\section{Hautes fréquences}
\section{Antennes}
\subsection{Dipôle}
2 brins de longueur totale $l$ et de diamètre $a$
$$l\ll \lambda$$
$$a\ll\lambda$$
Il existe deux zones
\begin{itemize}
\item Zone de Fresnel (zone de champ proche)
\item Zone de Fraunhofer (zone de champ lointain)
\end{itemize}
Il est important de se placer dans la bonne zone lorsqu'on fait des tests.
\subsection{Réflexion}
$$\Gamma=\frac{Z_{in}-Z_0}{Z_{in}-Z_0}$$
$$S_{11}=20\log_{10}(\Gamma)\quad [\si{\deci\bel}]$$
\subsection{Rayonnement}
On utilise la formule de Friis
$$P_r=P_tG_rG_t\left(\frac{\lambda}{4\pi R}\right)^2$$
en dB :
$$P_r=P_t+G_r+G_t+20\log\left(\frac{\lambda}{4\pi R}\right)$$
\subsection{Puissance}
$$\boxed{P_{\si{\deci\bel}}=10\log_{10}\left(P_{\si{\watt}}\right)}$$
$$\boxed{P_{\si{\deci\bel m}}=10\log_{10}\left(P_{\si{\milli\watt}}\right)}$$
\subsection{Zones}
\paragraph{Zone de Fresnel} $r_1 < x < r_2$
\paragraph{Zone de Fraunhofer} $ r_2 < x$

$$r_1=\sqrt{0.38\frac{D^3}{\lambda}}$$
$$r_2=\frac{2D^2}{\lambda}$$
\subsection{Remarques}
La différence entre le gain réalisé et le gain IEEE, le gain IEEE ne prend pas en compte d'éventuelles désadaptation de l'antenne (Hypothèse théorique d'une antenne parfaitement accordée). Le gain réalisé lui prend en compte cela (mesure réelle)





\end{document}